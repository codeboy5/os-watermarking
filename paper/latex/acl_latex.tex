% This must be in the first 5 lines to tell arXiv to use pdfLaTeX, which is strongly recommended.
\pdfoutput=1
% In particular, the hyperref package requires pdfLaTeX in order to break URLs across lines.

\documentclass[11pt]{article}

% Change "review" to "final" to generate the final (sometimes called camera-ready) version.
% Change to "preprint" to generate a non-anonymous version with page numbers.
\usepackage[review]{acl}

% Standard package includes
\usepackage{times}
\usepackage{latexsym}
\usepackage{amssymb}
\usepackage{amsmath}
\usepackage{amsthm}
\newtheorem{theorem}{Theorem}

% For proper rendering and hyphenation of words containing Latin characters (including in bib files)
\usepackage[T1]{fontenc}
% For Vietnamese characters
% \usepackage[T5]{fontenc}
% See https://www.latex-project.org/help/documentation/encguide.pdf for other character sets

% This assumes your files are encoded as UTF8
\usepackage[utf8]{inputenc}

% This is not strictly necessary, and may be commented out,
% but it will improve the layout of the manuscript,
% and will typically save some space.
\usepackage{microtype}

% This is also not strictly necessary, and may be commented out.
% However, it will improve the aesthetics of text in
% the typewriter font.
\usepackage{inconsolata}

%Including images in your LaTeX document requires adding
%additional package(s)
\usepackage{graphicx}

% If the title and author information does not fit in the area allocated, uncomment the following
%
%\setlength\titlebox{<dim>}
%
% and set <dim> to something 5cm or larger.

\title{Watermarking Open-Source LLMs with Context-Aware Logit Biasing}

% Author information can be set in various styles:
% For several authors from the same institution:
% \author{Author 1 \and ... \and Author n \\
%         Address line \\ ... \\ Address line}
% if the names do not fit well on one line use
%         Author 1 \\ {\bf Author 2} \\ ... \\ {\bf Author n} \\
% For authors from different institutions:
% \author{Author 1 \\ Address line \\  ... \\ Address line
%         \And  ... \And
%         Author n \\ Address line \\ ... \\ Address line}
% To start a separate ``row'' of authors use \AND, as in
% \author{Author 1 \\ Address line \\  ... \\ Address line
%         \AND
%         Author 2 \\ Address line \\ ... \\ Address line \And
%         Author 3 \\ Address line \\ ... \\ Address line}

\author{Miroojin Bakshi \\
  \texttt{miroojinb@iisc.ac.in} \\\And
  Danish Pruthi \\
  \texttt{danishp@iisc.ac.in} \\}

%\author{
%  \textbf{First Author\textsuperscript{1}},
%  \textbf{Second Author\textsuperscript{1,2}},
%  \textbf{Third T. Author\textsuperscript{1}},
%  \textbf{Fourth Author\textsuperscript{1}},
%\\
%  \textbf{Fifth Author\textsuperscript{1,2}},
%  \textbf{Sixth Author\textsuperscript{1}},
%  \textbf{Seventh Author\textsuperscript{1}},
%  \textbf{Eighth Author \textsuperscript{1,2,3,4}},
%\\
%  \textbf{Ninth Author\textsuperscript{1}},
%  \textbf{Tenth Author\textsuperscript{1}},
%  \textbf{Eleventh E. Author\textsuperscript{1,2,3,4,5}},
%  \textbf{Twelfth Author\textsuperscript{1}},
%\\
%  \textbf{Thirteenth Author\textsuperscript{3}},
%  \textbf{Fourteenth F. Author\textsuperscript{2,4}},
%  \textbf{Fifteenth Author\textsuperscript{1}},
%  \textbf{Sixteenth Author\textsuperscript{1}},
%\\
%  \textbf{Seventeenth S. Author\textsuperscript{4,5}},
%  \textbf{Eighteenth Author\textsuperscript{3,4}},
%  \textbf{Nineteenth N. Author\textsuperscript{2,5}},
%  \textbf{Twentieth Author\textsuperscript{1}}
%\\
%\\
%  \textsuperscript{1}Affiliation 1,
%  \textsuperscript{2}Affiliation 2,
%  \textsuperscript{3}Affiliation 3,
%  \textsuperscript{4}Affiliation 4,
%  \textsuperscript{5}Affiliation 5
%\\
%  \small{
%    \textbf{Correspondence:} \href{mailto:email@domain}{email@domain}
%  }
%}

\begin{document}
\maketitle

%%% Abstract %%%
\begin{abstract}
With the rise of human-like text generation by large language models (LLMs), reliably attributing machine-generated content has become increasingly important for combating misinformation, ensuring content provenance, and enforcing responsible AI usage. Watermarking, which embeds identifiable statistical signals in the generated text, offers a promising solution. However, traditional watermarking approaches assume exclusive control over the generation pipeline, typically embedding signals by modifying the text sampling process during inference. This assumption breaks down in the open-source setting, where models are fully accessible and users have unrestricted control over how text is generated. Existing open-weight watermarking methods often require costly tuning strategies, such as reinforcement learning, distillation from watermarked text, or supervised fine-tuning. In this work, we propose a method that directly modifies the unembedding layer through a structured perturbation conditioned on the model’s hidden states, in order to steer generation toward watermarked outputs. Our approach requires no retraining and integrates much faster than tuning-based methods, without sacrificing text fluency or watermark detectability. We conduct comprehensive evaluations on real-world, popular benchmarks and demonstrate that our watermarked model maintains strong performance. In addition, we show robustness to paraphrasing attacks and resilience to post-hoc model finetuning, establishing the practicality and effectiveness of our approach for watermarking open-source language models.
\end{abstract}


%%% Introduction %%%
\section{Introduction}

Watermarking techniques aim to embed imperceptible signals into model-generated text to enable provenance detection and mitigate misuse. While most existing approaches operate at decoding time by modifying next-token probabilities \citep{kirchenbauer2023watermark, aaronson2023reform, kuditipudi2023robust, liu2024adaptive}, these methods can be easily bypassed in open-weight models, where users have full control over the generation process. This motivates a shift toward techniques that embed the watermarking logic directly into the model's weights to produce detectable artifacts in the output distribution.

Existing approaches typically rely on distillation from watermarked teachers, finetuning, or reinforcement learning to steer generation toward watermarked outputs \citep{gu2023learnability, xu2024learningwatermarkllmgeneratedtext, elhassan2025can}. These tuning-based methods are computationally expensive, involve complex training pipelines, and struggle to maintain watermark detectability under low-distortion configurations.

Edit-based watermarking offers a lightweight alternative by modifying select model weights without retraining. \citet{christ2024provably} introduces a Gaussian watermark via an added output bias, but this non-standard component can be trivially removed. \textsc{GaussMark} \cite{block2025gaussmark} avoids architectural changes by perturbing existing weights and detecting watermarks via a gradient-based z-score, but suffers from weak detection signals and requires both forward and backward passes, limiting practicality.

In this work, we introduce a new \emph{edit-based watermarking framework} that modifies the unembedding layer weights—the parameters that project final hidden states to output vocabulary logits—by adding structured perturbations. These perturbations induce dynamic logit biases during generation that subtly influence token sampling in a detectable way. Unlike prior logit-based watermarking strategies such as \citet{kirchenbauer2023watermark} and \citet{liu2024adaptive}, which bias logit values during decoding, our approach directly embeds the biasing logic into the unembedding layer's weights. This design ensures that the watermark cannot be easily removed or circumvented, even when users have full control over the decoding process or inference code. Moreover, our framework is general-purpose: it supports multiple watermarking schemes that produce different biasing patterns, all of which are compatible with a unified detection method.

To detect the watermark, we apply a simple test: the \textbf{average log-likelihood ratio (LLR)} per token between the watermarked model and a reference model. This test captures subtle but consistent shifts in token probabilities introduced by the watermark. Crucially, it only requires forward passes and works across multiple watermarking schemes within our framework, provided they produce detectable logit perturbations.

We evaluate our method on three popular open-source LLMs, Llama-2-7b \citet{touvron2023llama}, Mistral-7B-v0.3 \citet{jiang2023mistral7b} and Qwen2.5-3B \citet{qwen2025qwen25technicalreport} under a range of conditions. Our results demonstrate that the proposed approach consistently achieves:

\begin{enumerate}

    \item Strong watermark detectability using the average LLR test

    \item Minimal impact on generation quality

\end{enumerate}

The rest of the paper is organized as follows: Section~\ref{sec:background_related} provides background on language model watermarking, with a focus on the unique challenges posed by open-source settings. Section~\ref{sec:methodology} introduces a general unembedding-layer watermarking framework, along with two concrete schemes and a corresponding statistical detection method. Section~\ref{sec:experiments} presents empirical evaluations of watermark detectability and text quality, including downstream task performance. Section~\ref{sec:paraphasing} examines robustness to paraphrasing attacks, and Section~\ref{sec:finetuning} investigates the durability of the watermark under model fine-tuning. Section~\ref{sec:conclusion} summarizes our findings and outlines directions for future work.

%%% Background and related Work %%%
\section{Background and Related Work}
\label{sec:background_related}

\subsection{Watermarking in Language Models}

Watermarking techniques aim to embed imperceptible signals into model-generated text to enable provenance detection and mitigate misuse. Most existing approaches operate at decoding time, modifying next-token probabilities based on pseudorandom functions over the vocabulary. A common strategy is to define a favored subset of tokens—the green list—by hashing the previous \(k\) tokens, and then bias generation toward these tokens using additive logit shifts or sampling-based constraints. For example, \citet{kirchenbauer2023watermark} apply soft logit biasing toward green-listed tokens, while \citet{aaronson2023reform} and \citet{kuditipudi2023robust} employ cryptographically driven scores to guide token sampling. Detection is typically performed through statistical tests that evaluate whether the distribution of generated tokens aligns with the expected green list patterns.

These methods typically involve a trade-off between two competing objectives: text generation quality and detection strength. Strong watermarking configurations—such as those using aggressive logit biasing or restrictive sampling—are easier to detect but tend to degrade text quality. In contrast, low-distortion watermarking setups preserve output quality but often result in signals that are harder to detect reliably.

\subsection{Open-Source Model Challenges}

While effective for API-served models, decoding-based watermarking fails in open-weight settings, where users can modify or bypass the decoding logic entirely. This motivates embedding the watermark directly into model weights. Model-level watermarking techniques fall into two categories:

\subsubsection{Tuning-based methods}

These methods fine-tune the model to produce watermarked outputs. For example, \citet{gu2023learnability} propose \emph{watermark distillation}, where a student model mimics the outputs of a decoding-time-watermarked teacher. While effective in principle, this approach is compute-intensive and brittle under post-hoc fine-tuning. Empirical evidence shows that low-distortion variants (e.g., KGW with small \(\delta\)) are particularly hard to learn and require significantly more training data to retain watermark detectability. 

\citet{xu2024learningwatermarkllmgeneratedtext} use reinforcement learning to embed watermarks by optimizing a composite reward balancing detectability and fluency. A paired detector is trained alongside the generator model to guide generation. While robust to paraphrasing and substitution, the method requires costly fine-tuning. 

\citet{elhassan2025can} introduce a dual-LoRA adapter framework where one model generates text and another evaluates it using the \emph{binoculars score}, a perplexity-based discrepancy metric. Training alternates between optimizing for utility and watermark strength using a regularized loss. While promising in detection accuracy, this method introduces co-training complexity and lacks evaluation under fine-tuning or transfer attacks.

\subsubsection{Edit-based methods} 
These approaches modify model weights post hoc without additional training.
\citet{christ2024provably} propose a watermarking method that adds small, known perturbations to the bias vector of the model's final layer. Detection involves summing the perturbations associated with the unique tokens in a generated text and checking whether the total exceeds a threshold. However, since most language models do not use a bias term in the final layer, this approach introduces a non-standard architectural component and the watermark can be trivially removed by deleting the added bias.

\citet{block2025gaussmark} extend noise-based watermarking by injecting Gaussian perturbations into a subset of model weights (typically in decoder layers). Detection computes the dot product between the noise vector and the gradient of the log-likelihood with respect to the perturbed weights. The method introduces significant limitations: it requires a forward and partial backward pass, and both detection and generation quality are highly sensitive to noise placement and strength.

\paragraph{Durability to Model Modifications.}
Durability is a central challenge in watermarking open-source models, which are frequently modified through quantization, pruning, merging, or fine-tuning. \citet{gloaguen2025towards} systematically evaluate existing approaches and find that none remain consistently detectable under such modifications.  In particular, distillation-based methods suffer from watermark decay under light supervision—even a few hundred steps of fine-tuning can erase the signal. Weight-editing schemes, while training-free, are often vulnerable to parameter shifts introduced by quantization or model merging. Their findings highlight the need for watermarking techniques designed explicitly with durability in mind.

\paragraph{Impact on downstream performance.}
Text quality is not the only metric affected by watermarking—\citet{ajith-etal-2024-downstream} show that even moderate-strength watermarks can significantly degrade performance on downstream tasks such as classification, QA, and generation. Since open-source watermarking methods embed the signal directly into model weights and cannot be easily undone once released, it is especially important to ensure that downstream task performance remains unaffected.




%%% Methodology %%%
\section{Methodology}
\label{sec:methodology}

This section introduces our watermarking method, which embeds structured perturbations into the unembedding matrix of a causal language model. We first define key notations, then describe the watermark construction and detection strategy, followed by our primary instantiation.

\subsection{Preliminaries}

Let a causal language model process a sequence of tokens drawn from a vocabulary \( \mathcal{V} = \{w_1, w_2, \dots, w_{|\mathcal{V}|}\} \). Let \( x_i \in \mathcal{V} \) denote the \( i \)-th token in the sequence, and let \( x_{<t} = (x_1, \dots, x_{t-1}) \) denote the prefix up to timestep \( t \).

The model computes a hidden representation \( h_t = f(x_{<t}) \in \mathbb{R}^d \), where \( f: \mathcal{V}^* \rightarrow \mathbb{R}^d \) is the model's internal encoding function. The logit vector \( v_t \in \mathbb{R}^{|\mathcal{V}|} \) for predicting the next token is given by:

\begin{equation}
    v_t = U h_t,
\end{equation}

where \( U \in \mathbb{R}^{|\mathcal{V}| \times d} \) is the unembedding matrix.

\subsection{Watermarking via Unembedding Perturbation}

We define a watermark by modifying the unembedding matrix with a structured perturbation:
\begin{equation}
    \tilde{U} = U + \Delta W,
\end{equation}
where \( \Delta W \in \mathbb{R}^{|V| \times d} \) is the watermarking matrix. This induces a modified logit vector:
\begin{equation}
    \tilde{v}_t = \tilde{U} h_t = v_t + \Delta W h_t,
\end{equation}
effectively applying a hidden-state-dependent logit bias during generation.

This general formulation enables a broad class of watermarking strategies. The structure of \( \Delta W \) governs the efficacy of the watermark. To be effective, \( \Delta W \) should produce logit biases that are:
\begin{itemize}
    \item \textbf{Detectable:} The logit biases should induce statistically identifiable changes in output probabilities.
    \item \textbf{Stealthy:} Perturbations should preserve fluency and perplexity of generated text, requiring logit biases to be in a reasonable range.
    \item \textbf{Diverse:} The logit biases should vary across contexts and timesteps, making it harder to reverse-engineer.
\end{itemize}
We describe two constructions of \( \Delta W \) that satisfy these criteria.


\subsubsection{Green List Biasing}

This method draws inspiration from the watermarking approach of \citet{kirchenbauer2023watermark}, which boosts the logits of tokens belonging to a pseudorandomly selected \emph{green list}. At each timestep, the green list is generated using a pseudorandom function (PRF) that depends on a secret seed and the preceding \( n \) tokens in the prefix. This ensures that the token biases vary in a deterministic, structured, and context-sensitive manner.

To encode this behavior directly into the model weights, we express the perturbation matrix \( \Delta W \) as the product of two matrices:
\begin{equation}
    \Delta W = G H,
\end{equation}
where:

\begin{itemize}
    \item \( G \in \mathbb{R}^{|\mathcal{V}| \times C} \) contains \( C \) watermarking lists represented as row vectors. For each pseudo-class \( c \in \{1, \dots, C\} \), the corresponding row \( H_c \in \mathbb{R}^{|\mathcal{V}|} \) is defined as:
    \[
    (H_c)_i = 
    \begin{cases}
    \delta & \text{if } i \in \mathcal{G}_c \\
    0 & \text{otherwise}
    \end{cases}
    \]
    where \( \delta > 0 \) is a fixed scalar that determines the strength of the watermark signal, and \( \mathcal{G}_c \subseteq \{1, \dots, |\mathcal{V}|\} \) is the green list for class \( c \), constructed using a pseudorandom function:
    \[
    \mathcal{G}_c = \left\{ i \in \{1, \dots, |\mathcal{V}|\} \;\middle|\; \mathrm{PRF}(\text{seed}, c, i) < \gamma \right\}.
    \]
    Here, \(\mathrm{PRF}(\cdot)\) is a keyed hash function seeded with a secret key, and \(\gamma \in (0,1)\) is a fixed threshold that determines the fraction of green-listed indices.

    \item \( H \in \mathbb{R}^{C \times d} \) maps each token's hidden state to a soft pseudo-class one-hot selector. To construct \( H \), we proceed as follows:
    \begin{enumerate}
        \item Run the base model on a large unlabeled corpus (e.g., OpenWebText) and collect hidden states from the last layer before the unembedding layer.
            \item Apply \( k \)-means clustering to these hidden states, assigning each to one of \( C \) pseudo-classes.
            \item Solve a ridge regression problem to map each hidden state \( h \in \mathbb{R}^d \) to a soft one-hot selector \( s \in \mathbb{R}^C \), approximating the discrete cluster assignments.
    \end{enumerate}
\end{itemize}

This construction ensures that hidden states in similar regions of representation space receive similar perturbations. Although the selectors in \( G \) are soft, they closely approximate discrete assignments and enable reliable watermark detection. The structured factorization of \( \Delta W \) into \( G \) and \( H \) allows multiple green list rules to be encoded in a single matrix perturbation.

\subsubsection{Gaussian Random Projection}

In this variant, the watermark is embedded by applying a fixed Gaussian perturbation to the unembedding matrix. Specifically, the perturbation matrix \( \Delta W \in \mathbb{R}^{|\mathcal{V}| \times d} \) is initialized as:
\begin{equation}
    \Delta W_{ij} \sim \mathcal{N}(0, 1).
\end{equation}

During generation, this produces a dynamic, input-dependent logit bias:
\[
\Delta \ell_t = \Delta W h_t \in \mathbb{R}^{|\mathcal{V}|},
\]
which is added to the model’s original logits.

Unlike the green list biasing approach, which relies on discrete vocabulary partitions, this method introduces a continuous watermarking signal by projecting the hidden state through a random Gaussian matrix. Although the hidden state \( h_t \in \mathbb{R}^d \) is not necessarily Gaussian, the Central Limit Theorem ensures that the projected logits \( [\Delta \ell_t]_i = \langle \Delta W_i, h_t \rangle \) are approximately Gaussian-distributed when \( d \) is large:

\begin{theorem}[CLT for Logit Bias Projection]
    Let \( h_t \in \mathbb{R}^d \) be fixed, and let each row \( \Delta W_i \in \mathbb{R}^d \) be drawn independently from \( \mathcal{N}(0, I_d) \). Then the scalar projection
    \[
        [\Delta W h_t]_i = \langle \Delta W_i, h_t \rangle = \sum_{j=1}^d \Delta W_{ij} \cdot h_{tj}
    \]
    converges in distribution to \( \mathcal{N}(0, \|h_t\|^2) \) as \( d \to \infty \).
\end{theorem}

Consequently, the logit bias vector \( \Delta \ell_t \) is approximately distributed as \( \mathcal{N}(0, \|h_t\|^2 I) \), meaning that, at each timestep, about half of the tokens are boosted while the other half are suppressed. This introduces a subtle but statistically detectable watermark signal without degrading generation quality.

To ensure consistent perturbation magnitude across timesteps and model scales, we normalize the matrix using the expected norm of a hidden state vector and introduce a scaling hyperparameter \( \delta > 0 \) to control watermark strength:
\begin{equation}
    \Delta W \leftarrow \frac{\delta}{\mathbb{E}[\|h_t\|]} \cdot \Delta W.
\end{equation}

\subsection{Detection via Likelihood Ratio Test}

To detect the watermark, we use a length-normalized log-likelihood ratio (LLR) test between the watermarked and reference models:
\begin{equation} \label{eq:llr}
    \text{LLR}(x) = \frac{1}{T} \sum_{t=1}^{T} \log \frac{p_{\text{wm}}(x_t \mid x_{<t})}{p_{\text{ref}}(x_t \mid x_{<t})},
\end{equation}
where \( p_{\text{wm}} \) and \( p_{\text{ref}} \) denote the softmax probabilities computed using the watermarked and original unembedding matrices, respectively.

These probabilities are defined as:
\begin{align}
    p_{\text{ref}}(x_t \mid x_{<t}) & =
    \frac{\exp(U_{x_t} h_t)}{\sum_{j=1}^{|V|} \exp(U_j h_t)}, \\
    p_{\text{wm}}(x_t \mid x_{<t})  & =
    \frac{\exp(\tilde{U}_{x_t} h_t)}{\sum_{j=1}^{|V|} \exp(\tilde{U}_j h_t)},
\end{align}
where \( U \in \mathbb{R}^{|V| \times d} \) is the original unembedding matrix, \( \tilde{U} = U + \Delta W \) is the watermarked version, and \( h_t \in \mathbb{R}^d \) is the hidden state at timestep \( t \).

Substituting these into Equation~\ref{eq:llr}, we obtain:
\begin{equation}
    \begin{aligned}
        \text{LLR}(x)
         & = \frac{1}{T} \sum_{t=1}^{T} \Big(
        (\tilde{U}_{x_t} - U_{x_t}) h_t                                       \\
         & \quad - \log \frac{\sum_j e^{\tilde{U}_j h_t}}{\sum_j e^{U_j h_t}}
        \Big)
    \end{aligned}
\end{equation}

This expression decomposes the LLR into two interpretable terms:
\begin{itemize}
    \item A \textit{token-level logit shift} term, \( (\tilde{U}_{x_t} - U_{x_t}) h_t \), which directly reflects the effect of watermarking on the predicted token's logit.
    \item A \textit{partition function ratio} term, which captures the normalization difference across the entire vocabulary.
\end{itemize}

% TODO:
% - Describe expected value of LLR under the null hypothesis (no watermark) and the alternative hypothesis (watermark present).
% - Discuss why a LLR is better than logit difference. Show rigorousness justification.

%%% Experiments %%%
\section{Experiments and Results}
\label{sec:experiments}

% - We evaluate our methods to assess the following:
%   - Detection accuracy for a given level of distortion
%       - We generate 500 samples watermarked text of length 200 tokens as continuations 50 token prompts from c4/realnewslike
%       - We compute AUCROC/Best F1 Score and TPR@0.1%FPR for each method
%       - We measure distortion using PPL computed by Llama2-13b as the oracle
%       - We compare with gaussmark (sigma=0.04, weights = model.layers.27.mlp.up_proj.weigh) and kgw logit distilled (delta=2.0, k=1, gamma=0.25) 
%       - We also compare with decoding based KGW as the upper bound baseline
%   - Downstream task performance
%       - We measure if there is any drop in performance on downstream tasks: hellaswag, gsm8k, and arc_challenge
%   - Robustness to paraphrasing
%       - We paraphrase our watermarked text using DIPPER with lexical diversity of 20 and 60
%       - We compute AUCROC/Best F1 Score and TPR@1%FPR for each method
%       - We compare with gaussmark (sigma=0.04, weights = model.layers.27.mlp.up_proj.weigh) and kgw logit distilled (delta=2.0, k=1, gamma=0.25)
%       - We also compare with decoding based KGW with k=0 as the upper bound baseline
%   


%%% Robustness to Paraphrasing %%%
\section{Robustness against Paraphrasing}
\label{sec:paraphasing}


%%% Durability against Fine-tuning %%%
\section{Durability of Watermarking to Fine-tuning}
\label{sec:finetuning}


%%% Conclusion %%%
\section{Conclusion}
\label{sec:conclusion}

\section{Preamble}

The first line of the file must be
\begin{quote}
\begin{verbatim}
\documentclass[11pt]{article}
\end{verbatim}
\end{quote}

To load the style file in the review version:
\begin{quote}
\begin{verbatim}
\usepackage[review]{acl}
\end{verbatim}
\end{quote}
For the final version, omit the \verb|review| option:
\begin{quote}
\begin{verbatim}
\usepackage{acl}
\end{verbatim}
\end{quote}

To use Times Roman, put the following in the preamble:
\begin{quote}
\begin{verbatim}
\usepackage{times}
\end{verbatim}
\end{quote}
(Alternatives like txfonts or newtx are also acceptable.)

Please see the \LaTeX{} source of this document for comments on other packages that may be useful.

Set the title and author using \verb|\title| and \verb|\author|. Within the author list, format multiple authors using \verb|\and| and \verb|\And| and \verb|\AND|; please see the \LaTeX{} source for examples.

By default, the box containing the title and author names is set to the minimum of 5 cm. If you need more space, include the following in the preamble:
\begin{quote}
\begin{verbatim}
\setlength\titlebox{<dim>}
\end{verbatim}
\end{quote}
where \verb|<dim>| is replaced with a length. Do not set this length smaller than 5 cm.

\section{Document Body}

\subsection{Footnotes}

Footnotes are inserted with the \verb|\footnote| command.\footnote{This is a footnote.}

\subsection{Tables and figures}

See Table~\ref{tab:accents} for an example of a table and its caption.
\textbf{Do not override the default caption sizes.}

\begin{table}
  \centering
  \begin{tabular}{lc}
    \hline
    \textbf{Command} & \textbf{Output} \\
    \hline
    \verb|{\"a}|     & {\"a}           \\
    \verb|{\^e}|     & {\^e}           \\
    \verb|{\`i}|     & {\`i}           \\
    \verb|{\.I}|     & {\.I}           \\
    \verb|{\o}|      & {\o}            \\
    \verb|{\'u}|     & {\'u}           \\
    \verb|{\aa}|     & {\aa}           \\\hline
  \end{tabular}
  \begin{tabular}{lc}
    \hline
    \textbf{Command} & \textbf{Output} \\
    \hline
    \verb|{\c c}|    & {\c c}          \\
    \verb|{\u g}|    & {\u g}          \\
    \verb|{\l}|      & {\l}            \\
    \verb|{\~n}|     & {\~n}           \\
    \verb|{\H o}|    & {\H o}          \\
    \verb|{\v r}|    & {\v r}          \\
    \verb|{\ss}|     & {\ss}           \\
    \hline
  \end{tabular}
  \caption{Example commands for accented characters, to be used in, \emph{e.g.}, Bib\TeX{} entries.}
  \label{tab:accents}
\end{table}

As much as possible, fonts in figures should conform
to the document fonts. See Figure~\ref{fig:experiments} for an example of a figure and its caption.

Using the \verb|graphicx| package graphics files can be included within figure
environment at an appropriate point within the text.
The \verb|graphicx| package supports various optional arguments to control the
appearance of the figure.
You must include it explicitly in the \LaTeX{} preamble (after the
\verb|\documentclass| declaration and before \verb|\begin{document}|) using
\verb|\usepackage{graphicx}|.

\begin{figure}[t]
  \includegraphics[width=\columnwidth]{example-image-golden}
  \caption{A figure with a caption that runs for more than one line.
    Example image is usually available through the \texttt{mwe} package
    without even mentioning it in the preamble.}
  \label{fig:experiments}
\end{figure}

\begin{figure*}[t]
  \includegraphics[width=0.48\linewidth]{example-image-a} \hfill
  \includegraphics[width=0.48\linewidth]{example-image-b}
  \caption {A minimal working example to demonstrate how to place
    two images side-by-side.}
\end{figure*}

\subsection{Hyperlinks}

Users of older versions of \LaTeX{} may encounter the following error during compilation:
\begin{quote}
\verb|\pdfendlink| ended up in different nesting level than \verb|\pdfstartlink|.
\end{quote}
This happens when pdf\LaTeX{} is used and a citation splits across a page boundary. The best way to fix this is to upgrade \LaTeX{} to 2018-12-01 or later.

\subsection{Citations}

\begin{table*}
  \centering
  \begin{tabular}{lll}
    \hline
    \textbf{Output}           & \textbf{natbib command} & \textbf{ACL only command} \\
    \hline
    \citep{Gusfield:97}       & \verb|\citep|           &                           \\
    \citealp{Gusfield:97}     & \verb|\citealp|         &                           \\
    \citet{Gusfield:97}       & \verb|\citet|           &                           \\
    \citeyearpar{Gusfield:97} & \verb|\citeyearpar|     &                           \\
    \citeposs{Gusfield:97}    &                         & \verb|\citeposs|          \\
    \hline
  \end{tabular}
  \caption{\label{citation-guide}
    Citation commands supported by the style file.
    The style is based on the natbib package and supports all natbib citation commands.
    It also supports commands defined in previous ACL style files for compatibility.
  }
\end{table*}

Table~\ref{citation-guide} shows the syntax supported by the style files.
We encourage you to use the natbib styles.
You can use the command \verb|\citet| (cite in text) to get ``author (year)'' citations, like this citation to a paper by \citet{Gusfield:97}.
You can use the command \verb|\citep| (cite in parentheses) to get ``(author, year)'' citations \citep{Gusfield:97}.
You can use the command \verb|\citealp| (alternative cite without parentheses) to get ``author, year'' citations, which is useful for using citations within parentheses (e.g. \citealp{Gusfield:97}).

A possessive citation can be made with the command \verb|\citeposs|.
This is not a standard natbib command, so it is generally not compatible
with other style files.

\subsection{References}


The \LaTeX{} and Bib\TeX{} style files provided roughly follow the American Psychological Association format.
If your own bib file is named \texttt{custom.bib}, then placing the following before any appendices in your \LaTeX{} file will generate the references section for you:
\begin{quote}
\begin{verbatim}
\bibliography{custom}
\end{verbatim}
\end{quote}

You can obtain the complete ACL Anthology as a Bib\TeX{} file from \url{https://aclweb.org/anthology/anthology.bib.gz}.
To include both the Anthology and your own .bib file, use the following instead of the above.
\begin{quote}
\begin{verbatim}
\bibliography{anthology,custom}
\end{verbatim}
\end{quote}

Please see Section~\ref{sec:bibtex} for information on preparing Bib\TeX{} files.

\subsection{Equations}

An example equation is shown below:
\begin{equation}
  \label{eq:example}
  A = \pi r^2
\end{equation}

Labels for equation numbers, sections, subsections, figures and tables
are all defined with the \verb|\label{label}| command and cross references
to them are made with the \verb|\ref{label}| command.

This an example cross-reference to Equation~\ref{eq:example}.

\subsection{Appendices}

Use \verb|\appendix| before any appendix section to switch the section numbering over to letters. See Appendix~\ref{sec:appendix} for an example.

\section{Bib\TeX{} Files}
\label{sec:bibtex}

Unicode cannot be used in Bib\TeX{} entries, and some ways of typing special characters can disrupt Bib\TeX's alphabetization. The recommended way of typing special characters is shown in Table~\ref{tab:accents}.

Please ensure that Bib\TeX{} records contain DOIs or URLs when possible, and for all the ACL materials that you reference.
Use the \verb|doi| field for DOIs and the \verb|url| field for URLs.
If a Bib\TeX{} entry has a URL or DOI field, the paper title in the references section will appear as a hyperlink to the paper, using the hyperref \LaTeX{} package.

\section*{Limitations}

Since December 2023, a "Limitations" section has been required for all papers submitted to ACL Rolling Review (ARR). This section should be placed at the end of the paper, before the references. The "Limitations" section (along with, optionally, a section for ethical considerations) may be up to one page and will not count toward the final page limit. Note that these files may be used by venues that do not rely on ARR so it is recommended to verify the requirement of a "Limitations" section and other criteria with the venue in question.

\section*{Acknowledgments}

This document has been adapted
by Steven Bethard, Ryan Cotterell and Rui Yan
from the instructions for earlier ACL and NAACL proceedings, including those for
ACL 2019 by Douwe Kiela and Ivan Vuli\'{c},
NAACL 2019 by Stephanie Lukin and Alla Roskovskaya,
ACL 2018 by Shay Cohen, Kevin Gimpel, and Wei Lu,
NAACL 2018 by Margaret Mitchell and Stephanie Lukin,
Bib\TeX{} suggestions for (NA)ACL 2017/2018 from Jason Eisner,
ACL 2017 by Dan Gildea and Min-Yen Kan,
NAACL 2017 by Margaret Mitchell,
ACL 2012 by Maggie Li and Michael White,
ACL 2010 by Jing-Shin Chang and Philipp Koehn,
ACL 2008 by Johanna D. Moore, Simone Teufel, James Allan, and Sadaoki Furui,
ACL 2005 by Hwee Tou Ng and Kemal Oflazer,
ACL 2002 by Eugene Charniak and Dekang Lin,
and earlier ACL and EACL formats written by several people, including
John Chen, Henry S. Thompson and Donald Walker.
Additional elements were taken from the formatting instructions of the \emph{International Joint Conference on Artificial Intelligence} and the \emph{Conference on Computer Vision and Pattern Recognition}.

% Bibliography entries for the entire Anthology, followed by custom entries
%\bibliography{anthology,custom}
% Custom bibliography entries only
\bibliography{custom}

\appendix

\section{Example Appendix}
\label{sec:appendix}

This is an appendix.

\end{document}