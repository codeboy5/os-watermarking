\section{Background and Related Work}
\label{sec:background_related}

\subsection{Watermarking in Language Models}

Watermarking techniques aim to embed subtle signals into model-generated text that are imperceptible to humans but detectable through statistical analysis. Most existing approaches operate at decoding time, modifying next-token probabilities based on pseudorandom functions that assign scores or preferences to tokens in the vocabulary. A common strategy is to define a favored subset of tokens—the green list—by hashing the previous \(k\) tokens, and then bias generation toward these tokens using additive logit shifts or sampling-based constraints. For example, \citet{kirchenbauer2023watermark} apply soft logit biasing toward green-listed tokens, while \citet{aaronson2023reform} and \citet{kuditipudi2023robust} employ cryptographically driven scores to guide token sampling. Detection is typically performed through statistical tests that evaluate whether the distribution of generated tokens aligns with the expected green list patterns.

These methods typically involve a trade-off between two competing objectives: text generation quality and detection strength. Strong watermarking configurations—such as those using aggressive logit biasing or restrictive sampling—are easier to detect but tend to degrade text quality. In contrast, low-distortion watermarking setups preserve output quality but often result in signals that are harder to detect reliably.

\subsection{Open-Source Model Challenges}

While effective for API-served models, decoding-based watermarking fails in open-weight settings, where users can modify or bypass the decoding logic entirely. This motivates embedding the watermark directly into model weights. Weight-based watermarking techniques fall into two categories:

\subsubsection{Tuning-based methods}

These methods fine-tune the model to produce watermarked outputs. For example, \citet{gu2023learnability} propose \emph{watermark distillation}, where a student model mimics the outputs of a decoding-time-watermarked teacher. While effective in principle, this approach is compute-intensive and brittle under post-hoc fine-tuning. Empirical evidence shows that low-distortion variants (e.g., KGW with small \(\delta\)) are particularly hard to learn and require significantly more training data to retain watermark detectability. 

\citet{xu2024learningwatermarkllmgeneratedtext} use reinforcement learning to embed watermarks by optimizing a composite reward balancing detectability and fluency. A paired detector is trained alongside the generator model to guide generation. While robust to paraphrasing and substitution, the method requires costly fine-tuning. 

\citet{elhassan2025can} introduce a dual-LoRA adapter framework where one model generates text and another evaluates it using the \emph{binoculars score}, a perplexity-based discrepancy metric. Training alternates between optimizing for utility and watermark strength using a regularized loss. While promising in detection accuracy, this method introduces co-training complexity and lacks evaluation under fine-tuning or transfer attacks.

\subsubsection{Edit-based methods} 
These approaches modify model weights post hoc without additional training.
\citet{christ2024provably} propose a watermarking method that adds small, known perturbations to the bias vector of the model's final layer. Detection involves summing the perturbations associated with the unique tokens in a generated text and checking whether the total exceeds a threshold. However, since most language models do not use a bias term in the final layer, this approach introduces a non-standard architectural component and the watermark can be trivially removed by deleting the added bias.

\citet{block2025gaussmark} extend noise-based watermarking by injecting Gaussian perturbations into a subset of model weights (typically in decoder layers). Detection computes the dot product between the noise vector and the gradient of the log-likelihood with respect to the perturbed weights. The method introduces significant limitations: it requires a forward and partial backward pass, and both detection and generation quality are highly sensitive to noise placement and strength.

\paragraph{Durability to Model Modifications.}
Durability is a central challenge in watermarking open-source models, which are frequently modified through quantization, pruning, merging, or fine-tuning. \citet{gloaguen2025towards} systematically evaluate existing approaches and find that none remain consistently detectable under such modifications.  In particular, distillation-based methods suffer from watermark decay under light supervision—even a few hundred steps of fine-tuning can erase the signal. Weight-editing schemes, while training-free, are often vulnerable to parameter shifts introduced by quantization or model merging. Their findings highlight the need for watermarking techniques designed explicitly with durability in mind.

\paragraph{Impact on downstream performance.}
Text quality is not the only metric affected by watermarking—\citet{ajith-etal-2024-downstream} show that even moderate-strength watermarks can significantly degrade performance on downstream tasks such as classification, QA, and generation. Since open-source watermarking methods embed the signal directly into model weights and cannot be easily undone once released, it is especially important to ensure that downstream task performance remains unaffected.


